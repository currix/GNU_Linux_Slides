\section{Text Editors}
%%%%%%%%%%%%%%%%%%%%%%%%%%%%%%%%%%%%%%%%%%%%%%%%%%%%%%%%%%%%%%%%%%%%%%%%%%%% 
%%%%%%%%%%%%%%%%%%%%%%%%%%%%%%%%%%%%%%%%%%%%%%%%%%%%%%%%%%%%%%%%%%%%%%%%%%%% 
\begin{frame}{What's a text editor?}{\emph{Writing is its own reward} -- Henry Miller}
  % -------------------------------------------------------
  
  \begin{block}{ \alert{Text editors} are used to create \emph{plain text} files}
    
    {\small
      \begin{itemize}
      \item They should not get mixed up with \alert{\textbf{word processors}}.
      \item Text editors only use characters in the selected character set, which normally is ASCII (American Standard Code for Information Interchange). 
      \item This is their main difference with (WYSIWYG) word processors that   generally include application specific control characters beyond the character set. 
      \item There is a large number of available text editors in \texttt{GNU/Linux} distributions.
      \end{itemize}
      }
  \end{block}
  
  \note{
    {\tiny
      
      Notes Module III
    }
  }
\end{frame}
%%%%%%%%%%%%%%%%%%%%%%%%%%%%%%%%%%%%%%%%%%%%%%%%%%%%%%%%%%%%%%%%%%%%%%%%%%%%%%%%%%%%%%%%%%%
%%%%%%%%%%%%%%%%%%%%%%%%%%%%%%%%%%%%%%%%%%%%%%%%%%%%%%%%%%%%%%%%%%%%%%%%%%%%%%%%%%%%%%%%%%%
\begin{frame}{Suggested \texttt{GNU/Linux} text editors}
  % -------------------------------------------------------
  
  \begin{block}{Some possible text editors options in \texttt{GNU/Linux}}
    
    {\small
      \begin{description}
      \item[emacs] {\footnotesize IMHO this is \emph{the editor}, either for
          preparing a \LaTeX document, a source code of a program (many
          plugins to facilitate programmer tasks) or to define macros to
          make complicate operations with your data files. You can even read
          mail, news or get help from a particular psychoanalist...}
      \item[vi/vim] {\footnotesize The original \texttt{UNIX} text editor (with \texttt{ed} permission). The main pro is that it can be found in any \texttt{UNIX} computer, the main con is a queer interface that can take some time to get used to (this point has improved with \texttt{vim}). It is good to know some basic \texttt{vi} usage for system administration.}
      \item[nano] {\footnotesize Simple and intuitive text editor that could replace \texttt{vi} for those that can not stand \texttt{vi} interface.}

      \item[Other options] {\footnotesize aee, jed, sublime, pluma, gedit, e3, eclipse, editra, elvis, fte, xemacs...} 
      \end{description}
     }
  \end{block}
  \note{
    {\tiny
      
      Notes Module III
    }
  }
\end{frame}
%%%%%%%%%%%%%%%%%%%%%%%%%%%%%%%%%%%%%%%%%%%%%%%%%%%%%%%%%%%%%%%%%%%%%%%%%%%%
%%%%%%%%%%%%%%%%%%%%%%%%%%%%%%%%%%%%%%%%%%%%%%%%%%%%%%%%%%%%%%%%%%%%%%%%%%%%
%%%%%%%%%%%%%%%%%%%%%%%%%%%%%%%%%%%%%%%%%%%%%%%%%%%%%%%%%%%%%%%%%%%%%%%%%%%% 
%%%%%%%%%%%%%%%%%%%%%%%%%%%%%%%%%%%%%%%%%%%%%%%%%%%%%%%%%%%%%%%%%%%%%%%%%%%%
\section{Basic Process Management}
%%%%%%%%%%%%%%%%%%%%%%%%%%%%%%%%%%%%%%%%%%%%%%%%%%%%%%%%%%%%%%%%%%%%%%%%%%%%
%%%%%%%%%%%%%%%%%%%%%%%%%%%%%%%%%%%%%%%%%%%%%%%%%%%%%%%%%%%%%%%%%%%%%%%%%%%%
\begin{frame}[t,fragile]{Checking System Processes}
  % ------------------------------------------------------


  \begin{block}{The \alert{\texttt{ps}} command}
    {\footnotesize
 Each of the tasks that a \texttt{UNIX} computer is processing
  at any moment is associated to what is called a \alert{process}
  with a unique \alert{PID} (Process IDentifier).

  The \texttt{ps} command displays current system processes. 

    }

  {\scriptsize

  \begin{columns}
      \column{.44\textwidth}The bareword \texttt{ps} command only displays the 
  processes associated to a session (discussed
  on next topic).

  Once a process is launched you can only pause or terminate it.
      \column{.56\textwidth}
        \hspace{-3cm}
%      {\scriptsize
        \begin{lstlisting}
alice@platea:~$ ps
  PID TTY          TIME CMD
 2530 pts/3    00:00:01 bash
 2626 pts/3    00:01:37 emacs
 2683 pts/3    00:00:09 evince
 2687 pts/3    00:00:12 evince
 3210 pts/3    00:00:00 ps
alice@platea:~$ 
        \end{lstlisting}
%      }
    \end{columns}
  }
  {\footnotesize
      The \alert{\texttt{ps}} command has many options (see
  \texttt{man ps}). To
  obtain a list in a long format of all the existing processes or
  tasks the appropriate options are \alert{\texttt{ps -aux}}. An alternative way
  of presenting process information is accomplished with the command
  \texttt{pstree} that emphasizes the relation among processes.
}
  \end{block}
  
%%%%%%%%%%%%%%%%%%%%%%%%%%%%%%%%%%%%%%%%%%%%%%%%%%%%%%%%%%%%%%%%%%%%%%%%%%%%%%%%%%
\note{
{\tiny

Notes Module III
}
}
\end{frame}
%%%%%%%%%%%%%%%%%%%%%%%%%%%%%%%%%%%%%%%%%%%%%%%%%%%%%%%%%%%%%%%%%%%%%%%%%%%%
%%%%%%%%%%%%%%%%%%%%%%%%%%%%%%%%%%%%%%%%%%%%%%%%%%%%%%%%%%%%%%%%%%%%%%%%%%%%
\begin{frame}[t,fragile]{Job Control Under \texttt{UNIX}}
  % ------------------------------------------------------


  \begin{block}{Sending processes to the background with  \alert{\texttt{\&}}}
    {\footnotesize
    If a program (e.g.\ a text editor) is launched from a console, the
console is inactive until the program finishes. This can be avoided if
the program is backgrounded, adding the \alert{\texttt{\&}} character at the end
of its activation statement.


      \begin{lstlisting}
alice@platea:~$  emacs filedit.txt &
      \end{lstlisting}
 
}
{\scriptsize

  \begin{columns}
      \column{.44\textwidth}A program in foreground can be
      stopped and backgrounded. The conventional stop key combination
      is \alert{\texttt{CTRL-Z}}
      and, once stopped, the program is
      backgrounded with the \alert{\texttt{bg}} command.  The command
      \alert{\texttt{jobs}} gives a list of backgrounded processes in a session.
      \column{.56\textwidth}
        \hspace{-3cm}
%      {\scriptsize
        \begin{lstlisting}
alice@platea:~$ xterm
^Z
[1]+  Stopped                 xterm
alice@platea:~$ bg
[1]+ xterm &
alice@platea:~$  jobs
[1]+  Running                 xterm &
alice@platea:~$
        \end{lstlisting}
%      }
    \end{columns}
  }
      {\footnotesize
In bash \alert{\texttt{\&}} is a command separator,  as  \alert{\texttt{;}}, but you can't use them simultaneously. 
}
  \end{block}
  
%%%%%%%%%%%%%%%%%%%%%%%%%%%%%%%%%%%%%%%%%%%%%%%%%%%%%%%%%%%%%%%%%%%%%%%%%%%%%%%%%%
\note{
{\tiny

Notes Module III
}
}
\end{frame}
%%%%%%%%%%%%%%%%%%%%%%%%%%%%%%%%%%%%%%%%%%%%%%%%%%%%%%%%%%%%%%%%%%%%%%%%%%%%
%%%%%%%%%%%%%%%%%%%%%%%%%%%%%%%%%%%%%%%%%%%%%%%%%%%%%%%%%%%%%%%%%%%%%%%%%%%%
\begin{frame}[t,fragile]{Killing processes}{\emph{Thou know'st 'tis common; all that lives must die...}, WS, Hamlet.}
  % ------------------------------------------------------

\vspace{-0.25cm}
  \begin{block}{Sending signals to processes with  \alert{\texttt{kill}}}
    {\footnotesize
The \alert{\texttt{kill}} command is intended, in general, to control
processes sending \emph{signals} to them, and in particular to terminate them. The default signal is \texttt{TERM}.


      \begin{lstlisting}
kill [ -signal | -s signal ] pid [...]
      \end{lstlisting}
 
}
{\scriptsize

  \begin{columns}
      \column{.44\textwidth} A program in foreground can
      be terminated with the key combination \alert{\texttt{CTRL-C}} or stopped
      with \alert{\texttt{CTRL-Z}}. These are two \texttt{kill}
      instances. A backgrounded program can be managed with the
      \texttt{kill} command.
      \column{.56\textwidth}
        \hspace{-3cm}
%      {\scriptsize
        \begin{lstlisting}
alice@platea:~$ sleep 1000 &
[1] 6254
alice@platea:~$ xterm &
[2] 6255
alice@platea:~$ ps
  PID TTY          TIME CMD
 6039 pts/3    00:00:00 bash
 6254 pts/3    00:00:00 sleep
 6255 pts/3    00:00:00 xterm
 6308 pts/3    00:00:00 ps
alice@platea:~$ kill 6254
[1]-  Terminated         sleep 1000
        \end{lstlisting}
%      }
    \end{columns}
  }
  \end{block}
  
%%%%%%%%%%%%%%%%%%%%%%%%%%%%%%%%%%%%%%%%%%%%%%%%%%%%%%%%%%%%%%%%%%%%%%%%%%%%%%%%%%
\note{
{\tiny

Notes Module III
}
}
\end{frame}
%%%%%%%%%%%%%%%%%%%%%%%%%%%%%%%%%%%%%%%%%%%%%%%%%%%%%%%%%%%%%%%%%%%%%%%%%%%%
%%%%%%%%%%%%%%%%%%%%%%%%%%%%%%%%%%%%%%%%%%%%%%%%%%%%%%%%%%%%%%%%%%%%%%%%%%%%
\begin{frame}[t,fragile]{Useful process signals}{Using the  \alert{\texttt{kill}} command}
  % ------------------------------------------------------
  \vspace{-0.35cm}
  \begin{block}{}
    {\footnotesize
You can check if you can send a signal to a process with the flag \alert{\texttt{-0}} and list the possible flags with \alert{\texttt{-L}} option.}
{\scriptsize
      \begin{lstlisting}
$ kill -0 1
bash: kill: (1) - Operation not permitted
$ kill -L
 1) SIGHUP	 2) SIGINT	 3) SIGQUIT	 4) SIGILL
      \end{lstlisting}
} 

{\scriptsize

  \begin{columns}
      \column{.4\textwidth} We are using in this example the \alert{\texttt{-19} (\texttt{SIGSTOP})}, \alert{\texttt{-18} (\texttt{SIGCONT})}, and \alert{\texttt{-15} (\texttt{SIGTERM})} signals. Another useful (though drastic) signal is \alert{\texttt{-9} (\texttt{SIGKILL})}.  
      \column{.6\textwidth}
        \hspace{-3cm}
%      {\scriptsize
        \begin{lstlisting}
$ ps
  PID TTY          TIME CMD
 1180 pts/0    00:00:00 bash
22898 pts/0    00:00:00 sleep
22913 pts/0    00:00:00 ps
$ kill -19 22898
[1]+  Stopped                 sleep 400 &
$ kill -18 22898
$ jobs
[1]   Running                 sleep 400 &
$ kill -15 22898
[1]+  Terminated              sleep 400 &
        \end{lstlisting}
%      }
    \end{columns}
  }
  \end{block}
  
%%%%%%%%%%%%%%%%%%%%%%%%%%%%%%%%%%%%%%%%%%%%%%%%%%%%%%%%%%%%%%%%%%%%%%%%%%%%%%%%%%
\note{
{\tiny
  If you want to run something in the background and then log out being able to reconnect later, use \alert{screen} (or \alert{tmux}). You can even attach multiple times, and each attached terminal will see (and control) the same thing.
  
The traditional approach still works: \texttt{nohup something \&}

Bash also has a \texttt{disown} command, if you want to log out with a background job running, and you forgot to nohup it initially.

\texttt{sleep 1000 [Ctrl-Z] bg disown}

   If you need to logout of an ssh session with background jobs still running, make sure their file descriptors have been redirected so they aren't holding the terminal open, or the ssh client may hang.
   
Notes Module III
}
}
\end{frame}
%%%%%%%%%%%%%%%%%%%%%%%%%%%%%%%%%%%%%%%%%%%%%%%%%%%%%%%%%%%%%%%%%%%%%%%%%%%% 
%%%%%%%%%%%%%%%%%%%%%%%%%%%%%%%%%%%%%%%%%%%%%%%%%%%%%%%%%%%%%%%%%%%%%%%%%%%%
\begin{frame}[t,fragile]{Keeping track of processes}
  % ------------------------------------------------------
  \begin{block}{The \alert{\texttt{top}} and \alert{\texttt{htop}} commands}
    {\footnotesize
 The command \alert{\texttt{top}}   provides  a dynamic real-time view of a running
       system. The previously described process management tasks can be performed
       using this program or its newer replacement \alert{\texttt{htop}}.}

{\tiny
         \begin{lstlisting}
           top - 16:51:06 up 18:23,  4 users,  load average: 0.15, 0.19, 0.12
Tasks: 164 total,   1 running, 163 sleeping,   0 stopped,   0 zombie
Cpu(s): 41.0%us, 8.1%sy, 0.0%ni, 50.3%id, 0.7%wa, 0.0%hi, 0.0%si, 0.0%st
Mem:   1016428k total,   918596k used,    97832k free,     9960k buffers
Swap:  1951740k total,   366252k used,  1585488k free,   381028k cached

  PID USER PR NI  VIRT  RES  SHR S %CPU %MEM    TIME+  COMMAND            
 1508 bob  20  0  308m  32m 9696 S  38  3.3  67:52.80 compiz             
 1627 bob  20  0 88036  18m 6628 S  13  1.8  17:43.08 unity-panel-ser    
  955 root 20  0  188m  13m 4072 S  12  1.4  27:09.33 Xorg               
 1447 bob  20  0  6944 2588  856 S  11  0.3  14:11.89 dbus-daemon        
 1698 bob  20  0 50700 1244  900 S   4  0.1   5:22.13 indicator-appli    
 1532 bob  20  0  216m  80m 9320 S   4  8.1   6:35.24 nm-applet          
 5211 bob  20  0  173m  17m   9m S   2  1.7  27:43.28 chrome             
      \end{lstlisting}
    } {\footnotesize Pressing \alert{\texttt{h}} you get help on
      \texttt{top}. To leave the program press \alert{\texttt{q}}. The
      \texttt{htop} command can be installed in \texttt{Mint}. It has
      a fancier and more modern interface with new features.

  }
  \end{block}
  
%%%%%%%%%%%%%%%%%%%%%%%%%%%%%%%%%%%%%%%%%%%%%%%%%%%%%%%%%%%%%%%%%%%%%%%%%%%%%%%%%%
\note{
{\tiny

Notes Module III
}
}
\end{frame}
%%%%%%%%%%%%%%%%%%%%%%%%%%%%%%%%%%%%%%%%%%%%%%%%%%%%%%%%%%%%%%%%%%%%%%%%%%%% 
%%%%%%%%%%%%%%%%%%%%%%%%%%%%%%%%%%%%%%%%%%%%%%%%%%%%%%%%%%%%%%%%%%%%%%%%%%%%
%%%%%%%%%%%%%%%%%%%%%%%%%%%%%%%%%%%%%%%%%%%%%%%%%%%%%%%%%%%%%%%%%%%%%%%%%%%% 
%%%%%%%%%%%%%%%%%%%%%%%%%%%%%%%%%%%%%%%%%%%%%%%%%%%%%%%%%%%%%%%%%%%%%%%%%%%%
\section{Redirection in Unix}
%%%%%%%%%%%%%%%%%%%%%%%%%%%%%%%%%%%%%%%%%%%%%%%%%%%%%%%%%%%%%%%%%%%%%%%%%%%%
%%%%%%%%%%%%%%%%%%%%%%%%%%%%%%%%%%%%%%%%%%%%%%%%%%%%%%%%%%%%%%%%%%%%%%%%%%%%
%%%%%%%%%%%%%%%%%%%%%%%%%%%%%%%%%%%%%%%%%%%%%%%%%%%%%%%%%%%%%%%%%%%%%%%%%%%%
%%%%%%%%%%%%%%%%%%%%%%%%%%%%%%%%%%%%%%%%%%%%%%%%%%%%%%%%%%%%%%%%%%%%%%%%%%%%
\begin{frame}[t,fragile]{Command Output Redirection}
  % ------------------------------------------------------
  \begin{block}{Redirecting \alert{\texttt{STDOUT}}}
    {\footnotesize Any command output is sent to the standard output
      (\texttt{STDOUT}) which by default is the screen. The output can be sent to a file using \alert{\texttt{>}} and
\alert{\texttt{>>}}, the redirection commands.

  Usage Examples
}

  {\scriptsize

  \begin{columns}
      \column{.44\textwidth} Saving the contents of the current directory into a file called \texttt{file\_list.txt}.
      \column{.56\textwidth}
        \hspace{-3cm}
      {\tiny
        \begin{lstlisting}
alice@platea:~$ ls
Downloads  nohup.out
alice@platea:~$ ls > file_list.txt
alice@platea:~$ ls
Downloads  file_list.txt nohup.out 
alice@platea:~$
        \end{lstlisting}
      }
    \end{columns}
  \begin{columns}
      \column{.44\textwidth}  A simple text file can be written combining the
  \texttt{cat} command and redirection.
      \column{.56\textwidth}
        \hspace{-3cm}
      {\tiny
        \begin{lstlisting}
alice@platea:~$ cat > output.txt
Write something here...
and press CTRL-D to exit
alice@platea:~$ ls o*txt
output.txt
        \end{lstlisting}
      }
    \end{columns}
  \begin{columns}
      \column{.44\textwidth}  The symbol \alert{\texttt{>>}} appends the command output to the end of the file, instead of creating a new file deleting the old one.
      \column{.56\textwidth}
        \hspace{-3cm}
      {\tiny
        \begin{lstlisting}
alice@platea:~$ cat l1.txt > out.txt
alice@platea:~$ cat l2.txt l3.txt >> out.txt
        \end{lstlisting}
      }
    \end{columns}
}
  \end{block}
  
%%%%%%%%%%%%%%%%%%%%%%%%%%%%%%%%%%%%%%%%%%%%%%%%%%%%%%%%%%%%%%%%%%%%%%%%%%%%%%%%%%
\note{
{\tiny

Notes Module III
}
}
\end{frame}
%%%%%%%%%%%%%%%%%%%%%%%%%%%%%%%%%%%%%%%%%%%%%%%%%%%%%%%%%%%%%%%%%%%%%%%%%%%%
%%%%%%%%%%%%%%%%%%%%%%%%%%%%%%%%%%%%%%%%%%%%%%%%%%%%%%%%%%%%%%%%%%%%%%%%%%%%
\begin{frame}[t,fragile]{Command Input Redirection}
  % ------------------------------------------------------


  \begin{block}{Redirecting  \alert{\texttt{STDIN}}}
    {\footnotesize
      Command input is loaded from \alert{\texttt{STDIN}}, the standard input, which by default is the keyboard. The standard input can be redefined using \alert{\texttt{<}}.


}
{\scriptsize

  \begin{columns}
      \column{.44\textwidth}This is an example of input redirection using the \texttt{sort} command. The file \texttt{list.txt} contains \texttt{\{ana, bob, alice, bara\}}.
      \column{.56\textwidth}
        \hspace{-3cm}
%      {\scriptsize
        \begin{lstlisting}
alice@platea:~$ sort < list.txt
alice
ana
bara
bob
alice@platea:~$
       \end{lstlisting}
%      }
    \end{columns}
  }
    {\footnotesize
       Input and output redirection can be combined.
{\scriptsize
      \begin{lstlisting}
alice@platea:~$ sort < list.txt > sorted.txt
alice@platea:~$ cat sorted.txt
alice
ana
bara
bob
      \end{lstlisting}
 }
}
  \end{block}
  
%%%%%%%%%%%%%%%%%%%%%%%%%%%%%%%%%%%%%%%%%%%%%%%%%%%%%%%%%%%%%%%%%%%%%%%%%%%%%%%%%%
\note{
{\tiny

Notes Module III
}
}
\end{frame}
%%%%%%%%%%%%%%%%%%%%%%%%%%%%%%%%%%%%%%%%%%%%%%%%%%%%%%%%%%%%%%%%%%%%%%%%%%%% 
%%%%%%%%%%%%%%%%%%%%%%%%%%%%%%%%%%%%%%%%%%%%%%%%%%%%%%%%%%%%%%%%%%%%%%%%%%%%
\begin{frame}[t,fragile]{Advanced Redirection}
  % ------------------------------------------------------

  \vspace{-0.3cm}
  \begin{block}{Managing \alert{\texttt{STDERR}}}
    {\footnotesize
 Apart from \alert{\texttt{STDIN}} -file descriptor (FD) 0- and \alert{\texttt{STDOUT}} -FD 1- also \alert{\texttt{STDERR}} -FD 2- can be considered. File descriptors are integer numbers that uniquely represents an opened file in an OS.

 \alert{\texttt{STDERR}} can be redefined using \alert{\texttt{2>}} while \alert{\texttt{1>}} corresponds to  \alert{\texttt{STDOUT}}.

  Usage Examples
}


\vspace{-0.3cm}
{\scriptsize

  \begin{columns}
      \column{.44\textwidth} This is an example of output redirection using the \texttt{ls} command.
      \column{.56\textwidth}
%      {\scriptsize
        \begin{lstlisting}
alice@platea:~$ ls > lsfile 2> err_lsfile
alice@platea:~$
        \end{lstlisting}
%      }
    \end{columns}
  \begin{columns}
    \column{.44\textwidth}  \texttt{STDOUT} and \texttt{STDERR} can be combined using the order \alert{\texttt{2>\&1}} like in the given examples. Multiple redirections are evaluated left to right, thus these two statements are not equivalent.
    \column{.56\textwidth}
%      {\scriptsize
        \begin{lstlisting}
alice@platea:~$ ls > lsfile 2>&1 
alice@platea:~$ ls 2>&1  > lsfile 
        \end{lstlisting}
%      }
    \end{columns}
  \begin{columns}
    \column{.44\textwidth} A shortcut to combine \texttt{STDOUT} and \texttt{STDERR} is available: \alert{\texttt{STDERR}}. Appending is possible too.
    \column{.56\textwidth}
%      {\scriptsize
        \begin{lstlisting}
alice@platea:~$ ls &> lsfile
alice@platea:~$ ls &>> lsfile
        \end{lstlisting}
%      }
    \end{columns}
}
  \end{block}
  
%%%%%%%%%%%%%%%%%%%%%%%%%%%%%%%%%%%%%%%%%%%%%%%%%%%%%%%%%%%%%%%%%%%%%%%%%%%%%%%%%%
\note{
{\tiny

Notes Module III
}
}
\end{frame}
%%%%%%%%%%%%%%%%%%%%%%%%%%%%%%%%%%%%%%%%%%%%%%%%%%%%%%%%%%%%%%%%%%%%%%%%%%%%
%%%%%%%%%%%%%%%%%%%%%%%%%%%%%%%%%%%%%%%%%%%%%%%%%%%%%%%%%%%%%%%%%%%%%%%%%%%%
%%%%%%%%%%%%%%%%%%%%%%%%%%%%%%%%%%%%%%%%%%%%%%%%%%%%%%%%%%%%%%%%%%%%%%%%%%%% 
%%%%%%%%%%%%%%%%%%%%%%%%%%%%%%%%%%%%%%%%%%%%%%%%%%%%%%%%%%%%%%%%%%%%%%%%%%%%
\section{Command Connection}
%%%%%%%%%%%%%%%%%%%%%%%%%%%%%%%%%%%%%%%%%%%%%%%%%%%%%%%%%%%%%%%%%%%%%%%%%%%%
%%%%%%%%%%%%%%%%%%%%%%%%%%%%%%%%%%%%%%%%%%%%%%%%%%%%%%%%%%%%%%%%%%%%%%%%%%%%
%%%%%%%%%%%%%%%%%%%%%%%%%%%%%%%%%%%%%%%%%%%%%%%%%%%%%%%%%%%%%%%%%%%%%%%%%%%%
%%%%%%%%%%%%%%%%%%%%%%%%%%%%%%%%%%%%%%%%%%%%%%%%%%%%%%%%%%%%%%%%%%%%%%%%%%%%
\begin{frame}[t,fragile]{Connecting commands output and input}
  % ------------------------------------------------------

  \vspace{-0.3cm}
  \begin{block}{Using UNIX pipes (\alert{\texttt{|}})}
    {\footnotesize
      The output of a command can be connnected
  (\emph{piped}) as the entry of a new command. This is achieved with
  a pipe, marked with a vertical bar character, \alert{\texttt{|}}.

  Usage Examples
}


\vspace{-0.3cm}
{\scriptsize
  \begin{columns}
      \column{.4\textwidth}Pipes allow to reduce the number of needed commands and avoid creating temporary files using pipes.
      \column{.6\textwidth}
        \begin{lstlisting}
alice@platea:~$ who | sort
alice pts/1  2012-03-17 19:30 (dynip.net)
bob   pts/18 2011-12-01 10:13 (tyrell)
root  pts/2  2011-09-16 12:56 (rachel)
root  pts/5  2011-12-22 13:36 (rick)
root  tty1   2012-01-27 12:57
alice@platea:~$ 
        \end{lstlisting}
    \end{columns}

    You can pipe several commands.

        \begin{lstlisting}
alice@platea:~$ ps aux | sort -r -k 5 |  less
USER   PID %CPU %MEM    VSZ   RSS TTY   STAT START  TIME COMMAND
root 16005 0.5  1.1  98024 47032 pts/1  Sl+  12:31  0:06 aptitude
alice 2697 0.0  0.1  84240  4328 ?      S    12:02  0:00 google-chrome
alice 2806 0.0  0.0   7816  3652 ?      S    12:02  0:00 xconsole 
root  2258 0.0  0.0   7772  3912 ?      Ss   12:02  0:00 NetworkManager
alice@platea:~$ 
        \end{lstlisting}
}
  \end{block}
  
%%%%%%%%%%%%%%%%%%%%%%%%%%%%%%%%%%%%%%%%%%%%%%%%%%%%%%%%%%%%%%%%%%%%%%%%%%%%%%%%%%
\note{
{\tiny

Notes Module III
}
}
\end{frame}
%%%%%%%%%%%%%%%%%%%%%%%%%%%%%%%%%%%%%%%%%%%%%%%%%%%%%%%%%%%%%%%%%%%%%%%%%%%%
%%%%%%%%%%%%%%%%%%%%%%%%%%%%%%%%%%%%%%%%%%%%%%%%%%%%%%%%%%%%%%%%%%%%%%%%%%%%
%%%%%%%%%%%%%%%%%%%%%%%%%%%%%%%%%%%%%%%%%%%%%%%%%%%%%%%%%%%%%%%%%%%%%%%%%%%%
%%%%%%%%%%%%%%%%%%%%%%%%%%%%%%%%%%%%%%%%%%%%%%%%%%%%%%%%%%%%%%%%%%%%%%%%%%%%
\section{Packaging and Compressing Files}
%%%%%%%%%%%%%%%%%%%%%%%%%%%%%%%%%%%%%%%%%%%%%%%%%%%%%%%%%%%%%%%%%%%%%%%%%%%%
%%%%%%%%%%%%%%%%%%%%%%%%%%%%%%%%%%%%%%%%%%%%%%%%%%%%%%%%%%%%%%%%%%%%%%%%%%%%
%%%%%%%%%%%%%%%%%%%%%%%%%%%%%%%%%%%%%%%%%%%%%%%%%%%%%%%%%%%%%%%%%%%%%%%%%%%%
%%%%%%%%%%%%%%%%%%%%%%%%%%%%%%%%%%%%%%%%%%%%%%%%%%%%%%%%%%%%%%%%%%%%%%%%%%%%
\begin{frame}[t,fragile]{File Compression and Expansion}
  % ------------------------------------------------------

  \vspace{-0.3cm}
  \begin{block}{Using \alert{\texttt{gzip}}/\alert{\texttt{gunzip}}}
    {\footnotesize Files have to be often compressed to save disk
  space. Depending on the file nature, the gaining of space can be
  huge. One of the most commonly used file compressors (decompressors)
  in \texttt{GNU/Linux} is \alert{\texttt{gzip}} (\alert{\texttt{gunzip}}). The standard suffix of files compressed with
  \texttt{gzip} is \alert{\texttt{.gz}}.

  Usage example  }



{\scriptsize
        \begin{lstlisting}
alice@platea:~/FSTS/tmp$ ls -lR /usr/share > f_1.txt
alice@platea:~/FSTS/tmp$ ls -lR /usr/lib > f_2.txt
alice@platea:~/FSTS/tmp$ ls -sh  f_*txt
 11M f_1.txt  1.8M f_2.txt
alice@platea:~/FSTS/tmp$ gzip f_*
alice@platea:~/FSTS/tmp$ ls -sh f_*
1.3M f_1.txt.gz  232K f_2.txt.gz
alice@platea:~$ 
        \end{lstlisting}


There are many other possible file compressors in
  \texttt{GNU/Linux}: \texttt{zip/unzip}, \texttt{rar/unrar},
  \texttt{bzip2/bunzip2}, etc...

}
  \end{block}
  
%%%%%%%%%%%%%%%%%%%%%%%%%%%%%%%%%%%%%%%%%%%%%%%%%%%%%%%%%%%%%%%%%%%%%%%%%%%%%%%%%%
\note{
{\tiny

Notes Module III
}
}
\end{frame}
%%%%%%%%%%%%%%%%%%%%%%%%%%%%%%%%%%%%%%%%%%%%%%%%%%%%%%%%%%%%%%%%%%%%%%%%%%%%
%%%%%%%%%%%%%%%%%%%%%%%%%%%%%%%%%%%%%%%%%%%%%%%%%%%%%%%%%%%%%%%%%%%%%%%%%%%%
\begin{frame}[t,fragile]{Creating a tarball with \alert{\texttt{tar}}}
  % ------------------------------------------------------

  \vspace{-0.3cm}
  \begin{block}{Packaging files with \alert{\texttt{tar}}}
    {\footnotesize Sometimes it is convenient to package several files and directories
  in a single file (a \emph{tarball} with suffix \alert{\texttt{tar}}) mostly for file sharing and
  transmission.

  The  \alert{\texttt{tar}} command syntax is \alert{\texttt{ tar cvf} \emph{file.tar} \texttt{file1 [file2 ...]}}.

   To check
  the contents of a tarball use the \alert{\texttt{tf}} options: \alert{\texttt{tar tf} file.tar} 

  Usage example  }



{\scriptsize
        \begin{lstlisting}
alice@platea:~/FSTS/tmp$ ls -lR /usr/share > file_1.txt
alice@platea:~/FSTS/tmp$ ls -lR /usr/lib > file_2.txt
alice@platea:~/FSTS/tmp$ tar cvf tar_test.tar file*txt 
file_1.txt
file_2.txt
alice@platea:~/FSTS/tmp$ ls -sh file* tar_test.tar 
11M file_1.txt  1.8M file_2.txt  13M tar_test.tar
alice@platea:~/FSTS/tmp$ tar tf tar_test.tar 
file_1.txt
file_2.txt
alice@platea:~/FSTS/tmp$
        \end{lstlisting}

}
  \end{block}
  
%%%%%%%%%%%%%%%%%%%%%%%%%%%%%%%%%%%%%%%%%%%%%%%%%%%%%%%%%%%%%%%%%%%%%%%%%%%%%%%%%%
\note{
{\tiny

Notes Module III
}
}
\end{frame}
%%%%%%%%%%%%%%%%%%%%%%%%%%%%%%%%%%%%%%%%%%%%%%%%%%%%%%%%%%%%%%%%%%%%%%%%%%%%
%%%%%%%%%%%%%%%%%%%%%%%%%%%%%%%%%%%%%%%%%%%%%%%%%%%%%%%%%%%%%%%%%%%%%%%%%%%%
\begin{frame}[t,fragile]{Extracting the tarball contents}
  % ------------------------------------------------------

  \vspace{-0.3cm}
  \begin{block}{Packaging files with \alert{\texttt{tar}}}
    {\footnotesize To extract files from a tarball the syntax is:
  \alert{\texttt{tar xvf}} \emph{file.tar} 

  Usage example  }



{\footnotesize
        \begin{lstlisting}
alice:~/FSTS/tmp$ mkdir testtar
alice:~/FSTS/tmp$ cd testtar
alice:~/FSTS/tmp/testtar$ tar xvf ../tar_test.tar 
file_1.txt
file_2.txt
alice:~/FSTS/tmp/testtar$ ls -hs
total 13M
 11M file_1.txt  1.8M file_2.txt
        \end{lstlisting}

}
{\footnotesize Remember that the files are unpacked to the current
  directory where the command \texttt{tar} is being executed.}
  \end{block}
  
%%%%%%%%%%%%%%%%%%%%%%%%%%%%%%%%%%%%%%%%%%%%%%%%%%%%%%%%%%%%%%%%%%%%%%%%%%%%%%%%%%
\note{
{\tiny

Notes Module III
}
}
\end{frame}
%%%%%%%%%%%%%%%%%%%%%%%%%%%%%%%%%%%%%%%%%%%%%%%%%%%%%%%%%%%%%%%%%%%%%%%%%%%%
%%%%%%%%%%%%%%%%%%%%%%%%%%%%%%%%%%%%%%%%%%%%%%%%%%%%%%%%%%%%%%%%%%%%%%%%%%%%
\begin{frame}[t,fragile]{Compressing and Packing in one go}
  % ------------------------------------------------------

  \vspace{-0.3cm}
  \begin{block}{Packaging and compressing files with \alert{\texttt{tar}}}
{\footnotesize
    Files can be simultaneously packaged and compressed. In
  such case the standard suffix is \alert{\texttt{.tar.gz}} or
  \alert{\texttt{.tgz}}. \\
To create a compressed tarball: \texttt{\alert{tar czvf}} \emph{file.tgz} \texttt{file1 [file2 ...]} \\
To check
  the contents of a tgz file: \texttt{\alert{tar tzf}}
  \emph{file.tgz}\\
To extract
  the contents of a tgz file: \texttt{\alert{tar xzvf}}
  \emph{file.tgz}
}
  
\vspace{0.2cm}
{\footnotesize
        \begin{lstlisting}
alice:~/FSTS/tmp$ tar czvf tar_test.tgz file*txt 
file_1.txt
file_2.txt
alice:~/FSTS/tmp$ ls -sh file* tar_test.tgz 
11M file_1.txt  1.8M file_2.txt  1.5M tar_test.tgz
alice:~/FSTS/tmp$ tar tzf tar_test.tgz 
file_1.txt
file_2.txt
alice:~/FSTS/tmp$

        \end{lstlisting}

}
  \end{block}
  
%%%%%%%%%%%%%%%%%%%%%%%%%%%%%%%%%%%%%%%%%%%%%%%%%%%%%%%%%%%%%%%%%%%%%%%%%%%%%%%%%%
\note{
{\tiny

Notes Module III
}
}
\end{frame}
%%%%%%%%%%%%%%%%%%%%%%%%%%%%%%%%%%%%%%%%%%%%%%%%%%%%%%%%%%%%%%%%%%%%%%%%%%%%
%%%%%%%%%%%%%%%%%%%%%%%%%%%%%%%%%%%%%%%%%%%%%%%%%%%%%%%%%%%%%%%%%%%%%%%%%%%%

