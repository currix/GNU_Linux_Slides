\section{Creating Folders}
%%%%%%%%%%%%%%%%%%%%%%%%%%%%%%%%%%%%%%%%%%%%%%%%%%%%%%%%%%%%%%%%%%%%%%%%%%%% 
%%%%%%%%%%%%%%%%%%%%%%%%%%%%%%%%%%%%%%%%%%%%%%%%%%%%%%%%%%%%%%%%%%%%%%%%%%%% 
\begin{frame}[fragile]{Creating directories with \alert{\texttt{mkdir}}}
  % -------------------------------------------------------
  
  \begin{block}{ Syntax: \alert{\texttt{mkdir}} \emph{folder\_name ...} }
    
  {\scriptsize
      \begin{lstlisting}
alice@platea:~$ ls
README  Desktop    Documents  Downloads  
alice@platea:~$
alice@platea:~$ mkdir tmp
alice@platea:~$ ls
README  Desktop    Documents  Downloads  tmp
alice@platea:~$ cd tmp
alice@platea:~/tmp$ 
      \end{lstlisting}
  }
    
    \begin{columns}[t]
      \column{.44\textwidth}
      {\footnotesize \emph{directory\_name} can be
        given as an \alert{absolute} or a \alert{relative} path. The following three commands are
        equivalent.}
      \column{.56\textwidth}
        \hspace{-3cm}
      {\scriptsize
        \begin{lstlisting}
alice@platea:~$ mkdir tmp
alice@platea:~$ mkdir /home/alice/tmp
alice@platea:~$ mkdir ./tmp
alice@platea:~$ mkdir ../alice/tmp
        \end{lstlisting}
      }
    \end{columns}

  \end{block}
  
  
  \note{
    {\tiny
      
      Notes Module I
    }
  }
\end{frame}
%%%%%%%%%%%%%%%%%%%%%%%%%%%%%%%%%%%%%%%%%%%%%%%%%%%%%%%%%%%%%%%%%%%%%%%%%%%%%%%%%%%%%%%%%%%
%%%%%%%%%%%%%%%%%%%%%%%%%%%%%%%%%%%%%%%%%%%%%%%%%%%%%%%%%%%%%%%%%%%%%%%%%%%%%%%%%%%%%%%%%%%
%%%%%%%%%%%%%%%%%%%%%%%%%%%%%%%%%%%%%%%%%%%%%%%%%%%%%%%%%%%%%%%%%%%%%%%%%%%% 
%%%%%%%%%%%%%%%%%%%%%%%%%%%%%%%%%%%%%%%%%%%%%%%%%%%%%%%%%%%%%%%%%%%%%%%%%%%%
\section{Copying Files or Folders}
%%%%%%%%%%%%%%%%%%%%%%%%%%%%%%%%%%%%%%%%%%%%%%%%%%%%%%%%%%%%%%%%%%%%%%%%%%%%
%%%%%%%%%%%%%%%%%%%%%%%%%%%%%%%%%%%%%%%%%%%%%%%%%%%%%%%%%%%%%%%%%%%%%%%%%%%%
\begin{frame}[fragile]{Copying files with \alert{\texttt{cp}}}
  % ------------------------------------------------------


  \begin{block}{The  \texttt{cp} command copies files and directories}
    {\footnotesize
      There are three possible ways of using this command:
    

      \begin{lstlisting}
cp [OPTION]... SOURCE DEST
cp [OPTION]... SOURCE... DIRECTORY
cp [OPTION]... -t DIRECTORY SOURCE...
      \end{lstlisting}
 

  Usage Examples
}

  {\scriptsize

  \begin{columns}[T]
      \column{.44\textwidth}Copy file \texttt{README} to
      \texttt{README.txt} in the same directory.
      \column{.56\textwidth}
        \hspace{-3cm}
%      {\scriptsize
        \begin{lstlisting}
alice@platea:~$ cp README README.txt
alice@platea:~$ 
        \end{lstlisting}
%      }
    \end{columns}
  \begin{columns}[T]
      \column{.44\textwidth} Copy files \texttt{README} and
      \texttt{README.2} to the user's home directory.
      \column{.56\textwidth}
        \hspace{-3cm}
%      {\scriptsize
        \begin{lstlisting}
alice@platea:/tmp$ cp README README.2 ~
alice@platea:/tmp$ 
        \end{lstlisting}
%      }
    \end{columns}
  \begin{columns}[T]
      \column{.44\textwidth} To copy a directory the option
    \texttt{-r} (recursively) should be used. The example shows how to
    copy the directory
    \texttt{\textasciitilde/Images} to the directory \texttt{./tmp}.
      \column{.56\textwidth}
        \hspace{-3cm}
%      {\scriptsize
        \begin{lstlisting}
alice@platea:~$ cp -r ~/Images ./tmp
alice@platea:~$
        \end{lstlisting}
%      }
    \end{columns}
}
  \end{block}
  
%%%%%%%%%%%%%%%%%%%%%%%%%%%%%%%%%%%%%%%%%%%%%%%%%%%%%%%%%%%%%%%%%%%%%%%%%%%%%%%%%%
\note{
{\tiny

Notes Module I
}
}
\end{frame}
%%%%%%%%%%%%%%%%%%%%%%%%%%%%%%%%%%%%%%%%%%%%%%%%%%%%%%%%%%%%%%%%%%%%%%%%%%%%
%%%%%%%%%%%%%%%%%%%%%%%%%%%%%%%%%%%%%%%%%%%%%%%%%%%%%%%%%%%%%%%%%%%%%%%%%%%%
%%%%%%%%%%%%%%%%%%%%%%%%%%%%%%%%%%%%%%%%%%%%%%%%%%%%%%%%%%%%%%%%%%%%%%%%%%%% 
%%%%%%%%%%%%%%%%%%%%%%%%%%%%%%%%%%%%%%%%%%%%%%%%%%%%%%%%%%%%%%%%%%%%%%%%%%%%
\section{Moving Files}
%%%%%%%%%%%%%%%%%%%%%%%%%%%%%%%%%%%%%%%%%%%%%%%%%%%%%%%%%%%%%%%%%%%%%%%%%%%%
%%%%%%%%%%%%%%%%%%%%%%%%%%%%%%%%%%%%%%%%%%%%%%%%%%%%%%%%%%%%%%%%%%%%%%%%%%%%
%%%%%%%%%%%%%%%%%%%%%%%%%%%%%%%%%%%%%%%%%%%%%%%%%%%%%%%%%%%%%%%%%%%%%%%%%%%%
%%%%%%%%%%%%%%%%%%%%%%%%%%%%%%%%%%%%%%%%%%%%%%%%%%%%%%%%%%%%%%%%%%%%%%%%%%%%
\begin{frame}[fragile]{Moving files with \alert{\texttt{mv}}}
  % ------------------------------------------------------


  \begin{block}{The  \texttt{mv} command moves files and directories}
    {\footnotesize
      There are three possible ways of using this command:
    

      \begin{lstlisting}
mv [OPTION]... SOURCE DEST
mv [OPTION]... SOURCE... DIRECTORY
mv [OPTION]... -t DIRECTORY SOURCE...
      \end{lstlisting}
 

  Usage Examples
}

  {\scriptsize

  \begin{columns}[T]
      \column{.44\textwidth} Rename file \texttt{README} to
      \texttt{README.txt} in the same directory.
      \column{.56\textwidth}
        \hspace{-3cm}
%      {\scriptsize
        \begin{lstlisting}
alice@platea:~$ mv README README.txt
alice@platea:~$ 
        \end{lstlisting}
%      }
    \end{columns}
  \begin{columns}[T]
      \column{.44\textwidth} Move files \texttt{README} and
      \texttt{README.2} to the user's home directory.
      \column{.56\textwidth}
        \hspace{-3cm}
%      {\scriptsize
        \begin{lstlisting}
alice@platea:/tmp$ mv README README.2 ~
alice@platea:/tmp$ 
        \end{lstlisting}
%      }
    \end{columns}
  \begin{columns}[T]
      \column{.44\textwidth} Move directory
    \texttt{\textasciitilde/Images} to directory \texttt{./tmp}.
      \column{.56\textwidth}
        \hspace{-3cm}
%      {\scriptsize
        \begin{lstlisting}
alice@platea:~$ mv ~/Images ./tmp
alice@platea:~$
        \end{lstlisting}
%      }
    \end{columns}
}
  \end{block}
  
%%%%%%%%%%%%%%%%%%%%%%%%%%%%%%%%%%%%%%%%%%%%%%%%%%%%%%%%%%%%%%%%%%%%%%%%%%%%%%%%%%
\note{
{\tiny

Notes Module I
}
}
\end{frame}
%%%%%%%%%%%%%%%%%%%%%%%%%%%%%%%%%%%%%%%%%%%%%%%%%%%%%%%%%%%%%%%%%%%%%%%%%%%%
%%%%%%%%%%%%%%%%%%%%%%%%%%%%%%%%%%%%%%%%%%%%%%%%%%%%%%%%%%%%%%%%%%%%%%%%%%%%
%%%%%%%%%%%%%%%%%%%%%%%%%%%%%%%%%%%%%%%%%%%%%%%%%%%%%%%%%%%%%%%%%%%%%%%%%%%% 
%%%%%%%%%%%%%%%%%%%%%%%%%%%%%%%%%%%%%%%%%%%%%%%%%%%%%%%%%%%%%%%%%%%%%%%%%%%%
\section{Deleting Files and Directories}
%%%%%%%%%%%%%%%%%%%%%%%%%%%%%%%%%%%%%%%%%%%%%%%%%%%%%%%%%%%%%%%%%%%%%%%%%%%%
%%%%%%%%%%%%%%%%%%%%%%%%%%%%%%%%%%%%%%%%%%%%%%%%%%%%%%%%%%%%%%%%%%%%%%%%%%%%
%%%%%%%%%%%%%%%%%%%%%%%%%%%%%%%%%%%%%%%%%%%%%%%%%%%%%%%%%%%%%%%%%%%%%%%%%%%%
%%%%%%%%%%%%%%%%%%%%%%%%%%%%%%%%%%%%%%%%%%%%%%%%%%%%%%%%%%%%%%%%%%%%%%%%%%%%
\begin{frame}[t,fragile]{Deleting folders with \alert{\texttt{rmdir}}}
  % ------------------------------------------------------

  \vspace{-0.3cm}
  \begin{block}{The  \texttt{rmdir} command  removes \emph{empty} directories}
    {\footnotesize
      \begin{lstlisting}
rmdir [OPTION]... DIRECTORY...
      \end{lstlisting}
 

  Usage Examples
}


\vspace{-0.3cm}
{\scriptsize

  \begin{columns}
      \column{.44\textwidth}  Create directories \texttt{READ0}
    and  \texttt{READ1} and remove them.
      \column{.56\textwidth}
%      {\scriptsize
        \begin{lstlisting}
alice@platea:~$ mkdir READ0 READ1
alice@platea:~$ ls -d READ0 READ1
READ0  READ1
alice@platea:~$ rmdir READ0 READ1
alice@platea:~$ ls -d READ0 READ1
ls: cannot access READ0: 
         No such file or directory
ls: cannot access READ1: 
          No such file or directory
        \end{lstlisting}
%      }
    \end{columns}
  \begin{columns}
      \column{.44\textwidth}  The \texttt{rmdir} command only
    removes empty folders.
      \column{.56\textwidth}
%      {\scriptsize
        \begin{lstlisting}
alice@platea:~$ mkdir READ0
alice@platea:~$ cp .bash_history READ0 
alice@platea:~$ rmdir READ0
rmdir: failed to remove `READ0/': 
               Directory not empty
alice@platea:~$
        \end{lstlisting}
%      }
    \end{columns}
}
  \end{block}
  
%%%%%%%%%%%%%%%%%%%%%%%%%%%%%%%%%%%%%%%%%%%%%%%%%%%%%%%%%%%%%%%%%%%%%%%%%%%%%%%%%%
\note{
{\tiny

Notes Module I
}
}
\end{frame}
%%%%%%%%%%%%%%%%%%%%%%%%%%%%%%%%%%%%%%%%%%%%%%%%%%%%%%%%%%%%%%%%%%%%%%%%%%%%
%%%%%%%%%%%%%%%%%%%%%%%%%%%%%%%%%%%%%%%%%%%%%%%%%%%%%%%%%%%%%%%%%%%%%%%%%%%%
\begin{frame}[t,fragile]{Deleting files (and folders) with \alert{\texttt{rm}}}
  % ------------------------------------------------------

  \vspace{-0.3cm}
  \begin{block}{The  \texttt{rm} command  and file deletion}
    {\footnotesize
The command \texttt{rm} removes files. By default it does not remove
directories. Be careful, there does not exist something like an
\textbf{undelete} command.

\begin{lstlisting}
rm [OPTION]... FILE...
\end{lstlisting}
 

  Usage Examples
}


\vspace{-0.3cm}
{\scriptsize

  \begin{columns}
      \column{.44\textwidth}  Remove files \texttt{README} and
      \texttt{README.txt}.
      \column{.56\textwidth}
%      {\scriptsize
        \begin{lstlisting}
alice@platea:~$ rm README README.txt
alice@platea:~$ 
        \end{lstlisting}
%      }
    \end{columns}
  \begin{columns}
      \column{.44\textwidth}  Frequently used options.
      \column{.56\textwidth}
%      {\scriptsize
        \begin{lstlisting}
-i   prompt before every removal
-f   ignore nonexistent files, never prompt
-r   remove directories and 
     their contents recursively
        \end{lstlisting}
%      }
    \end{columns}
  }
  {\footnotesize
   It is important to be very
    careful when using \alert{\texttt{rm -r}} as it is easy to wipe out
    important files or even an entire system.\\ \emph{With great power comes
    great responsability}...}
  \end{block}
  
%%%%%%%%%%%%%%%%%%%%%%%%%%%%%%%%%%%%%%%%%%%%%%%%%%%%%%%%%%%%%%%%%%%%%%%%%%%%%%%%%%
\note{
{\tiny

Notes Module I
}
}
\end{frame}
%%%%%%%%%%%%%%%%%%%%%%%%%%%%%%%%%%%%%%%%%%%%%%%%%%%%%%%%%%%%%%%%%%%%%%%%%%%%
%%%%%%%%%%%%%%%%%%%%%%%%%%%%%%%%%%%%%%%%%%%%%%%%%%%%%%%%%%%%%%%%%%%%%%%%%%%%
\begin{frame}{Saving keystrokes...}{\emph{Programmers can be lazy.} Larry Wall}
%-------------------------------------------------------
  \vspace{-0.3cm}
  \begin{block}{Saving on console typing}
    {\footnotesize Several key combinations save on typing, in particular \texttt{bash-completion} ``magic'' key \alert{\texttt{TAB}}.
    
    
    Usage Examples:
  }
  
{  \scriptsize
  \begin{itemize}
\item The \texttt{TAB} key is a very useful help. It autocompletes any command or file name, if there is only one possible option,  or gives a list of possible options. (``magic'' key).\pause
\item The command \texttt{history} provides a list of typed commands.\pause
\item The user can navigate in the \texttt{history} record with the \texttt{up}-arrow (also \texttt{CTRL-p}) and the \texttt{down}-arrow (also \texttt{CTRL-n}).\pause
\item Selecting text, the key combinations \texttt{Ctrl-Insert} or \texttt{Shift-Ctrl-c} (copy) and \texttt{Shift-Insert} or \texttt{Shift-Ctrl-v} (paste).\pause
\item Selecting text in a console and clicking on the second mouse button yanks this text in the prompt.
\end{itemize}

}  
  \end{block}
%%%%%%%%%%%%%%%%%%%%%%%%%%%%%%%%%%%%%%%%%%%%%%%%%%%%%%%%%%%%%%%%%%%%%%%%%%%%%%%%%%
\note{
{\tiny

Notes Module I
}
}
\end{frame}
%%%%%%%%%%%%%%%%%%%%%%%%%%%%%%%%%%%%%%%%%%%%%%%%%%%%%%%%%%%%%%%%%%%%%%%%%%%%
%%%%%%%%%%%%%%%%%%%%%%%%%%%%%%%%%%%%%%%%%%%%%%%%%%%%%%%%%%%%%%%%%%%%%%%%%%%%
\begin{frame}[t,fragile]{Using \alert{globbing}}
  % ------------------------------------------------------

  \vspace{-0.3cm}
  \begin{block}{File wildcards}
    {\footnotesize
File wildcards save typing, affecting several files simultaneously. The character \alert{\texttt{*}} matches any string, of any length.

\begin{tabular}{|l|l|}
\hline
\texttt{*} & Matches any string, of any length \\
\texttt{bla*} & Matches any string beginning with bla \\
\texttt{*x*} & Matches any string with an \texttt{x} at any position. \\
\texttt{*dat} & Matches any string finishing with \texttt{dat} \\
\hline
\end{tabular}

  Usage Examples
}


\vspace{-0.3cm}
{\scriptsize

  \begin{columns}
      \column{.44\textwidth} Move files ending in \texttt{.txt} in the working directory to the \texttt{/tmp} directory.
      \column{.56\textwidth}
%      {\scriptsize
        \begin{lstlisting}
alice@platea:~$ mv ./*.txt /tmp
alice@platea:~$ 
        \end{lstlisting}
%      }
    \end{columns}
  \begin{columns}
      \column{.44\textwidth}   Copy files beginning with \texttt{output} and ending with \texttt{.dat} to the user's home subdirectory \texttt{Results}.
      \column{.56\textwidth}
%      {\scriptsize
        \begin{lstlisting}
alice@platea:~$ cp output*.dat ~/Results
alice@platea:~$
        \end{lstlisting}
%      }
    \end{columns}
  \begin{columns}
      \column{.44\textwidth}  Delete all files in the current directory beginning with \texttt{a} and ending with \texttt{.ps}.
      \column{.56\textwidth}
%      {\scriptsize
        \begin{lstlisting}
alice@platea:~$ rm ./a*.ps
alice@platea:~$
        \end{lstlisting}
%      }
    \end{columns}
  }
  \end{block}
  
%%%%%%%%%%%%%%%%%%%%%%%%%%%%%%%%%%%%%%%%%%%%%%%%%%%%%%%%%%%%%%%%%%%%%%%%%%%%%%%%%%
\note{
{\tiny

Notes Module I
}
}
\end{frame}
%%%%%%%%%%%%%%%%%%%%%%%%%%%%%%%%%%%%%%%%%%%%%%%%%%%%%%%%%%%%%%%%%%%%%%%%%%%%
%%%%%%%%%%%%%%%%%%%%%%%%%%%%%%%%%%%%%%%%%%%%%%%%%%%%%%%%%%%%%%%%%%%%%%%%%%%%
\begin{frame}[t,fragile]{Using spaces and special characters in file names}
  % ------------------------------------------------------

  \vspace{-0.3cm}
  \begin{block}{Spaces and special characters}
    {\footnotesize
The use of spaces in filenames is allowed, though it is discouraged. Special characters use is even more controversial. Take into account that \alert{space} is the separator between arguments in \texttt{UNIX}.

Files differing on the use of upper and lowercase letters are \alert{different}.



  Usage Examples
}


%\vspace{-0.3cm}
{\scriptsize

  \begin{columns}
      \column{.3\textwidth}Remove a file called \texttt{``my notes.txt''}.
      \column{.7\textwidth}
%      {\scriptsize
        \begin{lstlisting}
alice@platea:~$ rm my\ notes.txt
alice@platea:~$ 
        \end{lstlisting}
%      }
    \end{columns}
  \begin{columns}
      \column{.3\textwidth}    Removing a file with spaces. Check the unexpected result.
      \column{.7\textwidth}
%      {\scriptsize
% alice@platea:~$ ls 
% alice@platea:~$ file.txt text.txt
      \begin{lstlisting}
alice@platea:~$ rm this is a text.txt
rm: cannot remove `this': No such file or directory
rm: cannot remove `is': No such file or directory
rm: cannot remove `a': No such file or directory
alice@platea:~$ rm ``this is a text.txt''
                OR
alice@platea:~$ rm this\ is\ a\ text.txt
alice@platea:~$
        \end{lstlisting}
%      }
    \end{columns}
  }
  \end{block}
  
%%%%%%%%%%%%%%%%%%%%%%%%%%%%%%%%%%%%%%%%%%%%%%%%%%%%%%%%%%%%%%%%%%%%%%%%%%%%%%%%%%
\note{
{\tiny

Notes Module I
}
}
\end{frame}
%%%%%%%%%%%%%%%%%%%%%%%%%%%%%%%%%%%%%%%%%%%%%%%%%%%%%%%%%%%%%%%%%%%%%%%%%%%%
%%%%%%%%%%%%%%%%%%%%%%%%%%%%%%%%%%%%%%%%%%%%%%%%%%%%%%%%%%%%%%%%%%%%%%%%%%%%
